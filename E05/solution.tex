% !TeX spellcheck = en_US
\documentclass[a4paper,twocolumn]{article}

\usepackage{fullpage}
\usepackage{fourier}
\usepackage{amsmath}
\usepackage{xcolor}
\usepackage{graphicx}
\usepackage{titlesec}
\usepackage{hyperref}
\usepackage{cleveref}
\usepackage{tabularx}

\titleformat{\subsection}[hang]{\large\bfseries}{\alph{subsection})\quad}{0pt}{}

\newcommand{\twodo}{\vspace{11pt}\textcolor{red}{\textbf{todo}}}
\newcommand{\subtask}[2]{\paragraph{#1)} \textit{#2} \newline}

\title{\textbf{Exercises for Image Processing 1}\\Problem Sheet 5}
\author{Axel Brand\\6145101 \and Nourhan Elfaramawy\\6517858 \and Sibel Toprak\\6712316}

\begin{document}
	\maketitle
	
	\section{Theoretical Problems}

	

	
	\subsection{(Lossless) Image Compression}
	
	\subsection{Image Segmentation}
	
	c. The need of component labeling for object detection after thresholding is due to the classification/distinguishment of pixels in each region of the image.
	
	d. The most common problem in edge detection aaproaches is its sensitvity to noise, which makes edge detection inaccurate.
	
	Canny-Algorithm:
	What Canny-algorithm does is that it takes the sobel operator and use is to thin the orientation of the edge, and then use the thresholding to find dominant edges in the image.
	The optimality of canny edge detection consists of the following criterion:
	1) The detection, which indicates all the important edges should not be missed.
	2) The localization, which minimizes the distance between the actual and the located position of the egdes.
	3) The one response, which minimizes multiple reponses to a single edge. Hence, reduces noise.
	
	
	\section{Practical Problems}
	
	\subsection{(Lossy) Image Compression}
	
	\subsection{Operators for Edge Detection}
	
\end{document}
