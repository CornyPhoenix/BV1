% !TeX spellcheck = en_US
\documentclass[a4paper,twocolumn]{article}

\usepackage{fullpage}
\usepackage{fourier}
\usepackage{amsmath}
\usepackage{xcolor}
\usepackage{graphicx}
\usepackage{titlesec}
\usepackage{hyperref}
\usepackage{cleveref}

\titleformat{\subsection}[hang]{\large\bfseries}{\alph{subsection})\quad}{0pt}{}

\newcommand{\twodo}[1]{\textcolor{red}{\textbf{todo:} #1}}
\newcommand{\subtask}[2]{\paragraph{#1)} \textit{#2} \newline}

\title{\textbf{Exercises for Image Processing 1}\\Problem Sheet 4}
\author{Axel Brand\\6145101 \and Nourhan Elfaramawy\\6517858 \and Sibel Toprak\\6712316}

\begin{document}
	\maketitle
	
	\section{Theoretical Problems}
	
	\subsection{Histogram and Noise}
	Histograms used to better understand how frequently or infrequently certain values occur in a given set of data. A greyvalue histogram of an image provides the frequency of greyvalues in the image. The histogram of an image with N quantization levels is represented by a 1D array with N elements. A greyvalue histogram describes discrete values.
	
	For an image showing white parsels on a black conveyor belt, I would expect clearly separated bimodality. In a histogram going from 0 (lightest) to 255 (darkest), one peak towards the 0 for the white parcels and the other one to the far side towards 255 for the black conveyor belt.
	
	If the camera is affected by sensor noise, this might change the histogram in that the two peaks cannot be clearly segregated, a valley occurs. The histogram appears to be somewhat blurred. It is difficult to determine a threshold by which we can decide whether the object in a scene is the parcel or the conveyor belt.
	
	\subsection{Projections}
	
	Create a projection of greyvalues in an image using column profile to get a column vector of all (normalized) row sums. The sum will be lower for rows showing parts of characters, while the lines going through will result in higher sums. 
	
	\subsection{Filters}
	
	\subsection{Convolution}
	
	Convolution describes filtering in the spatial domain. The Convolution Matrix  filter uses a  first matrix which is the Image to be treated. The image is a bi-dimensional collection of pixels in rectangular coordinates. The used kernel depends on task or the desired effect. We take in consideration a 3x3 Mask. The  filter studies successively every pixel of the image. For each of them, which we will call the  initial pixel, it multiplies the value of this pixel and values of the 8 surrounding pixels by the kernel corresponding value. Then it sums the results, and the initial pixel is set to this  final result value. When making the convolution of an image with itself, it means that the convolution matrix is the size of the image, giving the weight of every pixel the value 1, so the value of every single pixel depends on the values of the pixels of the entire image, which will result in a more blurred and fuzzy optical effect from the original image.
	
	\section{Practical Problems}
	
	\subsection{Greyvalue Normalization}
	
	\subsection{Fourier-Transform}
	
\end{document}
