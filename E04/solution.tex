% !TeX spellcheck = en_US
\documentclass[a4paper,twocolumn]{article}

\usepackage{fullpage}
\usepackage{fourier}
\usepackage{amsmath}
\usepackage{xcolor}
\usepackage{graphicx}
\usepackage{titlesec}
\usepackage{hyperref}
\usepackage{cleveref}

\titleformat{\subsection}[hang]{\large\bfseries}{\alph{subsection})\quad}{0pt}{}

\newcommand{\twodo}[1]{\textcolor{red}{\textbf{todo:} #1}}
\newcommand{\subtask}[2]{\paragraph{#1)} \textit{#2} \newline}

\title{\textbf{Exercises for Image Processing 1}\\Problem Sheet 4}
\author{Axel Brand\\6145101 \and Nourhan Elfaramawy\\6517858 \and Sibel Toprak\\6712316}

\begin{document}
	\maketitle
	
	\section{Theoretical Problems}
	
	\subsection{Histogram and Noise}
	
	A histogram is a graphical representation that visualizes how many times certain values occur in some given data. The data could be for example an image, in which case the corresponding color histogram represents the color distribution as it would plot the number of pixels for each color value. Conventionally, the horizontal axis represents the values, while the vertical axis represents their number of occurrence.
	
	For an image showing white parcels on a black conveyor belt, the kind of histogram to expect is a gray-value histogram, which provides the discrete gray-value distribution within the image. More particularly, the distribution of the gray-values would be a bimodal one with two more or less clearly distinguishable peaks: For gray-values ranging from 0 (lightest) to 255 (darkest), there would be one peak within a range around 0 for the white parcels and another one within a range around 255 for the black conveyor belt.
	
	If the camera is affected by sensor noise, the histogram might change in that the peaks appear to be shifted or the strict bi-modality is lost: The gray-values are distributed more evenly such that the peaks cannot be clearly distinguished anymore. It is difficult to determine a threshold by which we can decide with high accuracy whether an area in the image corresponds to a part of the parcel or of the conveyor belt.
	
	\subsection{Projections}
		
	To separate the text in the image into lines, one could project the gray-values in an image in ``column profile''. This yields a column vector of row sums. The row sum will be way higher for rows showing parts of characters compared to the sum for rows in the image that are in-between text lines. By using an appropriate value as threshold, we can detect the text lines in the image.
	
	The extracted image strips showing a text line can then be processed further to separate the characters. To that end, a gray-value projection is performed on each image strip, but this time in ``row profile''. Using the resulting row vector of column sums, we can distinguish between columns that are involved in the rendering of the characters and those that show white spaces: The column sum for the former is way larger. Based on that, the characters can be extracted from these image strips using an appropriate threshold value.
	
	
	
	\subsection{Filters}
	
	\subsection{Convolution}
	
	Convolution describes filtering in the spatial domain. The Convolution Matrix  filter uses a  first matrix which is the Image to be treated. The image is a bi-dimensional collection of pixels in rectangular coordinates. The used kernel depends on task or the desired effect. We take in consideration a 3x3 Mask. The  filter studies successively every pixel of the image. For each of them, which we will call the  initial pixel, it multiplies the value of this pixel and values of the 8 surrounding pixels by the kernel corresponding value. Then it sums the results, and the initial pixel is set to this  final result value. When making the convolution of an image with itself, it means that the convolution matrix is the size of the image, giving the weight of every pixel the value 1, so the value of every single pixel depends on the values of the pixels of the entire image, which will result in a more blurred and fuzzy optical effect from the original image.
	
	\section{Practical Problems}
	
	\subsection{Greyvalue Normalization}
	
	See \texttt{task\_2a.py}.
	
	\subsection{Fourier-Transform}
	
\end{document}
