% !TeX spellcheck = en_US
\documentclass[a4paper,12pt]{article}

\usepackage{fullpage}
\usepackage[utf8]{inputenc}
\usepackage{fourier}
\usepackage{amsmath}
\usepackage{color}
\usepackage{graphicx}
\usepackage{titlesec}

\titleformat{\subsection}[hang]{\large\bfseries}{\alph{subsection})\quad}{0pt}{}

\newcommand{\twodo}[1]{\textcolor{red}{\textbf{todo:} #1}}

\title{\textbf{Exercises for Image Processing 1}\\Problem Sheet 4}
\author{Tim Dobert\\6427948 \and Konstantin M\"ollers\\6313136}

\begin{document}
	\maketitle	
	
	\section{Theoretical Problems}
	\subsection{Histograms and Noise}
	
	\begin{itemize}
		\item Ein Histogramm repräsentiert, wie viele Pixel einen bestimmten Grauwert haben
		\item Das Histogramm hätte zwei Spitzen, eine für das Fließband, die andere für die Pakete
		\item Wenn das Rauschen Gaussverteilt ist, sollte sich das Histogramm eines normalen Bildes nicht großartig ändern.
		\item Für jeden Pixel der dunkler wird, wird ein anderer Heller.
		\item Außnahme ist ein komplett einfarbiges Bild
	\end{itemize}
	
	\subsection{Projections}
	\subsection{Filters}
	\subsection{Convolution}
	
	\section{Practical Problems}
	\subsection{Gray Value Normalization}
	\subsection{Fourier-Transform}
	
	1D DFT:
	
	\begin{align*}
		\hat{a}_l &= \sum\limits_{n = 0}^{N - 1} a_n \cdot e^{-2 \pi i \frac{nl}{N}} \\
		\hat{a}_k &= \sum\limits_{m = 0}^{M - 1} a_m \cdot e^{-2 \pi i \frac{mk}{M}} \\
	\end{align*}
	
	2D DFT:
	
	\begin{align*}
	\hat{a}_{kl} &= \sum\limits_{m = 0}^{M - 1} \sum\limits_{n = 0}^{N - 1} a_{m, n} \cdot e^{-2 \pi i \frac{mk}{M}} \cdot e^{-2 \pi i \frac{nl}{N}} \\
	\end{align*}
\end{document}
